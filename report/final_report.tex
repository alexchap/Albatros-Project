\documentclass[a4paper,english]{IEEEtran}
\usepackage[T1]{fontenc}
\usepackage[latin9]{inputenc}
\usepackage{babel}
\usepackage{graphicx}

\usepackage[unicode=true, pdfusetitle,
 bookmarks=true,bookmarksnumbered=false,bookmarksopen=false,
 breaklinks=false,pdfborder={0 0 1},backref=false,colorlinks=false]
 {hyperref}
\begin{document}

\date{December 2011}


\author{Alexandre Chappuis, Bastian Marquis and David Klopfenstein @ EPFL.ch}


\title{Implementation of a BGP Route Flap Damping Algorithm for the Bird
Routing Project}
\maketitle
\begin{abstract}

Today's Internet stability strongly relies on the good behavior of
dynamic routing protocols such as BGP (Border Gateway Protocol),
that enables routing between Autonomous Systems. Route flapping is
a well-known and undesirable phenomena occuring in both commercial
and private networks. In this report, we carefully explain our implementation
of the RFC 2439, BGP Route Flap Damping, for one famous open source
routing software suite, the Bird Routing Project.

\end{abstract}

\section{Introduction}

The inter-domain routing protocol BGP is still surviving to the gigantic
growth of the Internet that started during the last decade.
However, some widely used applications, such as Skype, still suffer from weaknesses
of that protocol.
The main problems are twofolds:
Firstly, BGP has a slow convergence rate, meaning that a change
at one location takes quite some time to be propagated throughout
the network. Secondly, if a node becomes unstable, for example if
its connectivity constantly comes up and down, it will have bad consequences
on the network, both in terms of useless processing at BGP routers and
unnecessary routing traffic. Routes advertised and withdrawn
at regular interval of time are said to be flapping.

Many approaches dealing with route flapping have been developped in the late 90's.
The RFC 2439\cite{rfc2439} was the first standard
proposed so that routers could inhibit the propagation of bad-behaving
routes, until they become stable again. People have used it extensively
for many years, in both commercial and open source routers.

Although this standard is not recommended anymore\cite{ripe recommendations}
in today's routers, we wanted to implement it for the Bird Routing
Project\cite{bird}, hoping that it will serve as a good basis for
future possible improvements and extension of this RFC. There exist
many variants of the Route Flap Damping alorithm and the community
has not lost its interest in finding robust mechanisms that could
allow BGP to be more resilient. 


\section{Overview}

RFC 2349 seeks to limit the impact of route flapping by ``damping'' (\textit{i.e.} ignoring packets of) missbehaving routes.
The solution must be able to distinguish flapping routes from good routes, consume few resources, both in terms of memory usage and process time.
The RFC solves these problems by assigning each route a penalty term.
Whenever this penalty term for a given route reaches a certain threshold, further advertisements for that route are ignored.
This penalty term varies over time : it is increased when the route becomes unreachable and decays as long as the route stays stable.
As soon as the figure of merit goes below a \textit{reuse threshold}, the route becomes again eligible for use.

The figure of merits decays exponentially over time.
Exponential decay has several advantages : it can be implemented very efficiently using precomputed \textit{decay arrays}.
Also, using exponential decay, the figure of merit keeps trace of previous instabilities for a fairly long time : old instabilities become less and less important over time, while newer ones have more weight.

Network administrators have lots of freedom in choosing the behavior of the penalty term.

Here is an example showing how the figure of merit evolve over time.
The route flapped four times before exceeding the \textit{cut threshold}.
It then remained stable.

The RFC proposes several optimizations to decrease processing time, at the cost of a slightly bigger memory footprint.

\section{Implementation}

this part is kind of straightforward -> just explain how we did it
with bird. cite coder's doc + github rep. of code


\subsection{Data structures}


\subsection{Processing withdrawals}


\subsection{Processing route advertisements}


\subsection{Configuration parameters}


\subsection{Timers}


\subsection{Miscellaneous}


\section{Evaluation}


\subsection{Small scale}

1 router with ca. 10 neighbors -> topology 3


\subsection{30 BGP routers in NSL cluster}

what we're going to need :
\begin{itemize}
\item numer of route damped for each bgp proto vs. number of diff. routes
advertised/withdrawn => show there are less updates and unnecessary
traffic when damping = activated
\item do it with different parameters ?
\item convergence time not affected ?
\end{itemize}

\section{Conclusion}

show importance of stability


\section{Future work}

possible extensions

real scale tests


\section{Acknowledgement}
\begin{thebibliography}{4}
\bibitem[1]{rfc2439}The RFC 2439, BGP Route Flap Damping, \href{http://www.ietf.org/rfc/rfc2439.txt}{http://www.ietf.org/rfc/rfc2439.txt}

\bibitem[2]{ripe recommendations} RIPE Recommendations On Route-flap
Damping, \href{http://www.ripe.net/ripe/docs/ripe-378}{http://www.ripe.net/ripe/docs/ripe-378}

\bibitem[3]{bird}Bird Routing Project, \href{http://bird.network.cz}{http://bird.network.cz}

\bibitem[4]{repository}Our publicly available repository, \href{https://github.com/alexchap/Albatros-Project}{https://github.com/alexchap/Albatros-Project}
\end{thebibliography}

\end{document}
